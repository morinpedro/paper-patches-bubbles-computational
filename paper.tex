\documentclass{article}
%\usepackage{t1enc}

\usepackage[T1]{fontenc}
% \renewcommand{\familydefault}{\sfdefault}%change letter
\usepackage[utf8]{inputenc}
\usepackage{amsfonts,mathtools,amssymb}
\usepackage{graphicx,wrapfig,tikz,animate,subcaption}
\usepackage{lmodern}
\usepackage{fancyhdr,graphicx,wrapfig,amsmath,amssymb, mathtools, scrextend, titlesec, enumitem}
\usepackage{hyperref}
\usepackage{caption}
% \usepackage{subcaption}
\usepackage{comment}
\usepackage{cancel}

\usetikzlibrary{arrows}

\pagestyle{plain}
% \setlength{\topmargin}{-2cm}
% \setlength{\textheight}{25cm}
% \setlength{\textwidth}{16cm}
% \setlength{\oddsidemargin}{-0.5cm}
% \setlength{\evensidemargin}{-1.5cm}
% \setlength{\parskip}{2mm}

\definecolor{verylightgray}{gray}{0.93}
\usepackage[color=verylightgray]{todonotes}

\usepackage{wrapfig,tikz}
\usetikzlibrary{shapes.misc}
% Entornos
\newtheorem{thm}{Teorema}
\newtheorem{coro}[thm]{Corollary}
\newtheorem{defi}[thm]{Definition}
\newtheorem{lemma}[thm]{Lemma}
\newtheorem{propo}[thm]{Proposition}
\newtheorem{obs}[thm]{Remark}
\newtheorem{rem}[thm]{Remark}
\usepackage{xspace}
%Genericas
\newcommand{\N}{\ensuremath{\mathbb{N}}\xspace}
\newcommand{\R}{\ensuremath{\mathbb{R}}\xspace}
\newcommand{\Q}{\ensuremath{\mathcal{Q}}\xspace}
\newcommand{\T}{\ensuremath{\mathcal{T}}\xspace}
\newcommand{\aaa}{\ensuremath{\mathbf{a}}\xspace}
\newcommand{\nnn}{\ensuremath{\mathbf{n}}\xspace}
\DeclareMathOperator{\Div}{div}
\newcommand{\tfa}{\text{for all }}
\newcommand{\tand}{\quad\text{and}\quad}
\newcommand{\Hoi}[1][\Omega]{\ensuremath{H_0^1(#1)}\xspace}
\newcommand{\bT}{\ensuremath{b_1^T}}
\newcommand{\tbT}{\ensuremath{\tilde b_1^T}}
\renewcommand{\theenumi}{\arabic{enumi}}

\title{Patch bubbles for convection-dominated problems}
\author{Eberhard B\"ansch \and Pedro Morin \and Itatí Zócola}
\date{}

\begin{document}

	\maketitle


\tableofcontents


\section{Introduction}
In this article we consider the following advection dominated advection-reaction-diffusion problem, in a domain $\Omega$ of $\R^d$, with $d\in\{2,3\}$:
\begin{equation*}
    \left\{
\begin{aligned}
    u_t - \epsilon \Delta u + \aaa\cdot \nabla u + c \, u &=0 ,\qquad& &\text{in $\Omega$, $t>0$,}\\
        u &= 0, \qquad& &\text{on $\partial\Omega$}, t>0,\\
    u(\cdot,0) &= u^0, \qquad& & \text{in }\Omega.\\
\end{aligned}
    \right.
\end{equation*}
Here
$f\in L^2(\Omega)$,
$u_0 \in L^2(\Omega)$, 
$\aaa \in W^{1,\infty}(\Omega)$ and $c\in L^\infty(\Omega)$, with 
$c-\frac12 \Div \aaa \le 0$.


\todo[inline]{last but not least 1}

\section{Residual Free Bubbles for Stationary Problems}

\todo[inline]{A couple of paragraphs about partitions and finite element spaces}

\todo[inline]{Some historical comments.}


\subsection{Element Bubbles}


In this section we recall some ideas from the paper~\cite{Brezzi1999}%
\todo{Ita, please check in~\cite{Brezzi1999} in which paper the residual free bubbles were first introduced. Is it paper [1] from this reference?}.

In what follows we use the notation from~\cite{Brezzi1999}.
They considered the following advection-diffusion problem:
\begin{equation}\label{eq:convdiff-stat-strong}
    \left\{
\begin{aligned}
    - \epsilon \Delta u + \aaa\cdot \nabla u &=f ,\qquad& &\text{in $\Omega$,}\\
        u &= 0, \qquad& &\text{on $\partial\Omega$}, 
\end{aligned}
    \right.
\end{equation}
with (piecewise) constant $\aaa$ and $f$.
The associated bilinear form is
\[
a(\psi,\varphi) = \int_\Omega \epsilon \nabla \psi \cdot \nabla \varphi + \aaa \cdot \nabla \psi \, \varphi.
\]
And the weak formulation of~\eqref{eq:convdiff-stat-strong} reads
\[
\text{Find } u \in \Hoi : \qquad a(u,\varphi) = (1,\varphi), \quad\tfa \varphi \in \Hoi,
\]
where $(f,\varphi)=\int_\Omega f\varphi$, as usual.

For a triangulation of simplices $\T$, and for each element $T\in \T$ they consider the \emph{residual-free bubble} $\bT$ defined as follows:
\[
\bT\in \Hoi[T]:\qquad a(\bT,\varphi)=(1,\varphi), \quad \tfa \varphi\in \Hoi[T].
\]
Whence, integration by parts leads to
\begin{equation}\label{eq:bT-identity-1}
\epsilon \| \nabla \bT \|_{0,T}^2  = \int_T \bT,
\end{equation}
and, for every affine function $\varphi$,
\begin{equation}\label{eq:bT-identity-2}
\int_T -\epsilon \nabla\bT \cdot\nabla \varphi = 0
\tand
\int_T \aaa \cdot \nabla \bT \, \varphi = - \int_T \aaa \cdot \nabla \varphi \, \bT
= -\aaa\cdot\nabla \varphi\int_T \bT.
\end{equation}

Then, for every $T\in\T$, they consider the one-dimensional space $B_T$ spanned by \bT and set 
\[
V_B = \bigoplus_{T\in\T}B_T
\tand
V_h = V_L + V_B.
\]
The discrete problem now reads
\[
\text{Find } u_h = u_L + u_B \in V_h:\qquad
a(u_h,v_h) = (f,v_h),\tfa v_h \in V_h.
\]
After testing separately for $v_h \in V_B$ and $v_h \in V_L$ they perform a static condensation and arrive at a linear system of equations for $u_L$, which is what practitioners are usually interested in.
In this respect, the use of bubbles can be regarded as a \emph{stabilization method} for the finite element solution $u_L$, and coincides with the famous SUPG method~\cite{BFHR1997}, with a precise definition of the stabilization constants.

Moreover, due to~\eqref{eq:bT-identity-1}--\eqref{eq:bT-identity-2}, the stabilization only requires the computation of $\int_T \bT$ and no other features of $\bT$.
When $\epsilon \ll h/|\aaa|$, this integral can be very well approximated by the integral of $\tbT$, which is the solution of the purely transport problem
\[
\aaa\cdot \nabla \tbT = 1, \ \text{in $T$}, 
\qquad
\bT = 0, \ \text{on $\partial_+ T$},
\]
with $\partial_+ T$ the \emph{inflow} part of $\partial T$ (where $\aaa \dot \nnn\le 0$).


\todo[inline]{Describe the usual element bubbles~\cite{Brezzi1999}, and the fact that for the computation only their integral is necessary. Say also that the numerical results do not look so satisfactory to us.}

\subsection{Patch Bubbles}

\todo[inline]{In order to add more \emph{freedom} to the space, we incorporate \emph{patch bubbles}. Start with our proposal, and mention also Cangiani-Süli's papers~\cite{Cangiani2005long,Cangiani2005short}.
Emphasize that the computation with these bubbles leads to the necessity of computing them with more resolution than with the element bubbles. Therefore we propose to use recursion.}

\subsection{Bubble Computation}

\todo[inline]{Present the recursive algorithm}

\subsection{Numerical Experiments}

\todo[inline]{Comparison between usual element bubbles and patch bubbles. Two or three experiments}

\section{Instationary Problems}

We now consider the instationary advection dominated advection-diffusion problem
\begin{equation*}
    \left\{
\begin{aligned}
    u_t - \epsilon \Delta u + \aaa\cdot \nabla u + c \, u &=0 ,\qquad& &\text{in $\Omega$, $t>0$,}\\
        u &= 0, \qquad& &\text{on $\partial\Omega$}, t>0,\\
    u(\cdot,0) &= u^0, \qquad& & \text{in }\Omega.\\
\end{aligned}
    \right.
\end{equation*}

\subsection{New method}

\todo[inline]{Emphasize on the simplicity of keeping the information from the bubbles, against the computation of the static condensation. Keeping the information from the bubbles also leads to a very stable and precise method.}

\subsection{Numerical Experiments}

\todo[inline]{Comparison between usual element bubbles and patch bubbles. Two or three experiments}

\section{Conclusions}

\todo[inline]{last but not least 2}

\bibliographystyle{abbrv}
\bibliography{biblioRFB}
\end{document}
