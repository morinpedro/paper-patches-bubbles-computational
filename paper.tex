\documentclass[11pt,a4paper]{article}
%\usepackage{t1enc}

\usepackage[T1]{fontenc}
% \renewcommand{\familydefault}{\sfdefault}%change letter
\usepackage[utf8]{inputenc}
\usepackage{amsfonts,mathtools,amssymb}
\usepackage{graphicx,wrapfig,tikz,animate,subcaption}
\usepackage{lmodern}
\usepackage{fancyhdr,graphicx,wrapfig,amsmath,amssymb, mathtools, scrextend, titlesec, enumitem}
\usepackage{hyperref}
\usepackage{caption}
% \usepackage{subcaption}
\usepackage{comment}
\usepackage{cancel}

\usetikzlibrary{arrows}

\pagestyle{plain}
% \setlength{\topmargin}{-2cm}
% \setlength{\textheight}{25cm}
% \setlength{\textwidth}{16cm}
% \setlength{\oddsidemargin}{-0.5cm}
% \setlength{\evensidemargin}{-1.5cm}
% \setlength{\parskip}{2mm}

\usepackage{todonotes}

\usepackage{wrapfig,tikz}
\usetikzlibrary{shapes.misc}
% Entornos
\newtheorem{thm}{Teorema}
\newtheorem{coro}[thm]{Corollary}
\newtheorem{defi}[thm]{Definition}
\newtheorem{lemma}[thm]{Lemma}
\newtheorem{propo}[thm]{Proposition}
\newtheorem{obs}[thm]{Remark}
\newtheorem{rem}[thm]{Remark}
%Genericas
\newcommand{\NN}{\ensuremath{\mathbb{N}}\hspace{0 cm}}
\newcommand{\RR}{\ensuremath{\mathbb{R}}\hspace{0 cm}}
\newcommand{\Q}{\ensuremath{\mathcal{Q}}\hspace{0 cm}}
\newcommand{\aaa}{\ensuremath{\mathbf{a}}\hspace{0 cm}}

\renewcommand{\theenumi}{\arabic{enumi}}

\title{Patch bubbles for convection-dominated problems}
\author{Eberhard B\"ansch \and Pedro Morin \and Itatí Zócola}
\date{}

\begin{document}

	\maketitle


\tableofcontents


\section{Introduction}

\todo[inline]{last but not least 1}

\section{Residual Free Bubbles for Stationary Problems}

We consider the following advection dominated advection-diffusion problem:
\begin{equation*}
    \left\{
\begin{aligned}
    - \epsilon \Delta u + \aaa\cdot \nabla u &=0 ,\qquad &\text{in $\Omega$,}\\
        u|_{\partial\Omega} &= 0.\\
\end{aligned}
    \right.
\end{equation*}


\subsection{Element Bubbles}

\todo[inline]{Describe the usual element bubbles~\cite{Brezzi1999}, and the fact that for the computation only their integral is necessary. Say also that the numerical results do not look so satisfactory to us.}

\subsection{Patch Bubbles}

\todo[inline]{In order to add more \emph{freedom} to the space, we incorporate \emph{patch bubbles}. Start with our proposal, and mention also Cangiani-Süli's.
Emphasize that the computation with these bubbles leads to the necessity of computing them with more resolution than with the element bubbles. Therefore we propose to use recursion.}

\subsection{Bubble Computation}

\todo[inline]{Present the recursive algorithm}

\subsection{Numerical Experiments}

\todo[inline]{Comparison between usual element bubbles and patch bubbles. Two or three experiments}

\section{Instationary Problems}

We now consider the instationary advection dominated advection-diffusion problem
\begin{equation*}
    \left\{
\begin{aligned}
    u_t - \epsilon \Delta u + \aaa\cdot \nabla u &=0 ,\qquad& &\text{in $\Omega$, $t>0$,}\\
        u|_{\partial\Omega} &= 0, \qquad& & t>0,\\
    u(\cdot,0) &= u^0, \qquad& & \text{in }\Omega.\\
\end{aligned}
    \right.
\end{equation*}

\subsection{New method}

\todo[inline]{Emphasize on the simplicity of keeping the information from the bubbles, against the computation of the static condensation, which also leads to a very stable and precise method.}

\subsection{Numerical Experiments}

\todo[inline]{Comparison between usual element bubbles and patch bubbles. Two or three experiments}

\section{Conclusions}

\todo[inline]{last but not least 2}

\bibliographystyle{abbrv}
\bibliography{biblioRFB}
\end{document}
